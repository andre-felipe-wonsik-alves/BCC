\begin{abstract}
Much is unknown about how living systems grow into, coordinate communication across, and maintain themselves as hierarchical arrangements of semi-independent cells, tissues, organs, and entire
bodies, where each component at each level has its own goals and sensor, motor, and communication capabilities. Similar uncertainty surrounds exactly how selection acts on the components across
these levels. Finally, growing interest in viewing intelligence not as something localized to the brain but rather distributed across biological hierarchies has renewed investigation into the nature
of such hierarchies. Here we show that organizing neural cellular automata (NCAs) into a hierarchical structure can improve the ability to evolve them to perform morphogenesis and homeostasis,
compared to non-hierarchical NCAs. The increased evolvability of hierarchical NCAs (HNCAs) compared to non-hierarchical NCAs suggests an evolutionary advantage to the formation and utilization
of higher-order structures, across larger spatial scales, for some tasks, and suggests new ways to design and optimize NCA models and hierarchical arrangements of robots. The results presented here
demonstrate the value of explicitly incorporating hierarchical structure into systems that must grow and maintain complex patterns. The introduced method may also serve as a platform to further
investigate the evolutionary dynamics of multiscale systems.
\end{abstract}

\keywords{cellular automata, neural cellular automata, morphogenesis, multiscale, complex systems, hierarchical}