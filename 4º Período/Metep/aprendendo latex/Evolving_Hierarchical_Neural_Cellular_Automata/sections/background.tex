\section{Background}
Communication within biological systems occurs across multiple
scales. Cells can communicate directly with one another through
“key and lock” mechanisms on their membrane surfaces; on a local scale by releasing signaling chemicals (paracrine factors) into
their local environment; or on vaster scales with endocrine and
bio-electric signaling. The variety of signaling modalities is constantly being expanded with new discoveries \cite{Cardozo2020,Levin2023,Mayr1997,Mordvintsev2021} and each
mode of communication empowers individual cells or higher-order
structures to influence and guide the state of the whole. Previous
NCA models of morphogenesis do not attempt to explicitly capture
these larger-scale communication modes. The work done in HCAs
does roughly capture this, though their evolutionary dynamics have
not been analyzed nor have they been used to test our multiscale
signaling hypothesis (to our knowledge).

Selection is one essential mechanism of evolution. There are
many perspectives within evolutionary biology regarding at which
level(s) selection occurs \cite{Goertzel1992,McMillenLevin2024,Palm2022}. Selection can only act on the
physical manifestation of an organism: its phenotype. However,
discrete phenotypes can be identified at multiple scales. There is
general agreement that behaviors at any of these scales may affect
fitness, but there’s disagreement about whether to think of this as
selection at many scales, or indirect influences that subtly affect
fitness at the scale of an organism \cite{Darwin1859}, or its genes \cite{Dawkins1976}. By applying
selection to different scales of an HNCA, we are able to investigate
how much, or whether, this influences the overall hierarchy’s ability
to grow and maintain structure.

\subsection{Related Work}
The recent NCA literature largely explores generative modeling, texture generation, and morphogenesis applications. The Variational
Neural Cellular Automata (VNCA) \cite{Pande2023} introduces a probabilistic
generative model inspired by cellular growth and differentiation,
demonstrating the ability to reconstruct and generate diverse outputs from a common vector format despite its simplicity and limited
parameters. Another study on texture generation with NCAs \cite{Nichele2018}
showcases their potential in learning to generate high-fidelity textures from single template images, though it fails consistently in
reproducing larger-scale structures. CA-NEAT \cite{Noble2008} utilizes compositional pattern producing networks (CPPNs) evolved with neuroevolution to model morphogenesis and replication. Grasso et. al
\cite{GrassoBongard2022} evolves single-resolution NCAs using an evolutionary algorithm
(EA) and analyzes each cell as individual agents with their own
sensorimotor loops.

HCAs were introduced by Weimar et. al [29] in 1998 to combine
mesoscopic and microscopic modeling details with application to
catalytic reaction on a surface. Kiester et. al \cite{KiesterSahr2008} explores a large-scale
multi-resolution CA with 3 levels of resolution, using an extensive
hand-designed ruleset. In 2010, Kroc et. al \cite{Dunn2010} reviewed hierarchical
CA methods, though optimization for a task was not discussed. Hierarchical neural CA models emerge roughly in 2017; Qin et. al \cite{SchmidtLipson2011}
propose HCAs which are a combination of cuboid CAs (CCAs) and
single-layer CAs (SCAs) into an HCA (though they do not use the
term HCA). In 2023, Pande et. al \cite{Qin2018} explores parent-child 2-scale
HNCA architectures, though did not investigate morphogenesis
and homeostasis dynamics. All of these HCA rulesets have either
been hand-designed or optimized with gradient descent; here we
optimize with an evolutionary algorithm (EA).

Evolving HNCAs may, in the future, let us examine the evolutionary dynamics of multiscale systems in general and could contribute
to validation of theories about Hierarchical selection \cite{Goertzel1992,Palm2022}. Hierarchical selection posits that selection occurs at multiple scales at
once. This could be explicitly modeled, evolved, and analyzed in
our system. The theory that evolution exploits hierarchical communication to avoid the intractable connection costs required to
coordinate large-scale structure in a flat system \cite{BullmoreSporns2012,MooreCao2008} might also
be probed with our approach.