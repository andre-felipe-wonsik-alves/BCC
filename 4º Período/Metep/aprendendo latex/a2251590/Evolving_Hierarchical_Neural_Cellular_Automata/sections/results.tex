{\section{Results}
Here we show that constraining a multi-resolution NCA into a
hierarchically integrative structure enables the system to achieve higher fitness on a shape-matching and homeostasis task than nonhierarchical NCAs. We also show that explicit selection for higherorder/lower-resolution structures can result in varying fitness for
similar tasks.
\Needspace{0.45\textheight}
\begin{strip}
  \centering
  \includegraphics[width=\linewidth]{assets/figures/comparations.png}
  \captionof{figure}{Evolvability of HNCA compared to the control (non-hierarchical NCAs). (a-c): average loss across 20 trials for 1-layer non-hierarchical NCA (blue), 2-layer, 3-layer, and 4-layer hierarchical NCAs (orange, green, red) selected for square (a), diamond (b), and hollow circle (c) shape-matching. (d-f): strip plots of the loss for the top-fitness NCAs of all 20 independent trials at generation 2000 (y-axis) for 1 (blue), 2 (orange), 3 (green), and 4-layer (red) NCAs. Each of (d-f) matches the shape of the (a-c) above it.}
  \label{fig:results_comparations}
\end{strip}

\subsection{Hierarchical structure achieves higher fitness}
Figure \ref{fig:results_comparations} shows the shape-matching ability of 1, 2, 3, and 4-layer
HNCAs where loss is evaluated at the lowest layer (64x64) for each
shape. Importantly, the comparisons in Figure \ref{fig:results_comparations} comparisons aren’t
quite fair - each additional level fundamentally extends the model,
adding to the model’s computational capacity. Thus we compare
the performance of HNCAs to analogous NCAs with equivalent
computational capacity which are not explicitly hierarchical to
determine the hierarchical structure’s impact on the fitness of these
systems. In Figure \ref{fig:hnca_vs_nca} we show that the HNCA outperforms the
control NCA in several cases

\subsection{Hierarchical selection}
To explore the idea of hierarchical selection, we evaluated the shapematching phenotypes of individual HNCAs at varying levels of
resolution. In these experiments, for each shape, we ran 4 separate
treatments where selection occurred at the 4 levels of resolution
previously discussed. Figure \ref{fig:target_shape} shows the target shapes for each
level of resolution for each of the 4 shapes.

In Figure \ref{fig:average_evolutionary}, the highest-fitness individuals which were selected
at the different levels of the hierarchy vary in their ability to perform the task at their level. For the square, the populations which
were selected at every level below the topmost level were more
fit than those whose phenotypes were evaluated at layer 4 (8x8
resolution). For the diamond shape, the results are similar. In both
of these cases (square and diamond), selection at the lowest layer
appears to perform best. In contrast, when matching to the hollow
circle, HNCAs selected at the highest level outperformed the HNCA
selected at the lower levels.