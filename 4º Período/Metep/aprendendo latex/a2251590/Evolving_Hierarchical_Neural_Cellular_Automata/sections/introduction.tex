\section{Introduction}
A growing view in cognitive science is that intelligence is not localized to the nervous system but rather arises from the hierarchical organization of living systems \cite{Mengistu2016}. In this view, each component at each level — organ, tissue, cell — has its own goals, sensor/motor capabilities, and ability to communicate across its level of organization, downward to components below it, and upward to components above. These different scales are thoroughly integrated into the whole organism, but structures at different scales may be said to be semi-autonomous agents in their own right, possessing their own intelligence and competency in their own problem space \cite{Lyon2006,Mathijssen2019,Mengistu2016,Okasha2012}.

Cellular automata (CA) are a natural choice for modeling growth and communication between large numbers of biological components, usually cells. This is because the fate of a biological cell during development is largely a function of its local interactions \cite{Edelman1989}. Evolving hierarchical cellular automata could provide a useful way for investigating the various ways of exerting selection pressure at higher or lower levels of organization. One can then observe what effect, if any, this has on the evolvability of the system as a whole, and its ability to grow and maintain complex patterns at one or more levels. Specifically, here we test the hypothesis that hierarchical cellular automata are more evolvable than those lacking hierarchy because evolution can exploit structures at larger scales to coordinate the growth and maintenance of patterns at smaller scales.

Herein we employ neural cellular automata \cite{GrassoBongard2022,Nichele2018,Noble2008,Pande2023}, as neural networks provide a more flexible method for updating the states of cells and thus for evolving CA rule sets. Evolving hierarchical NCAs has to our knowledge never been done before and we think HNCAs could serve as a viable platform for exploring multiscale evolutionary dynamics. 

Here we attempt to capture, in a simple model, the spatially hierarchical aspect of biological systems. We combine modeling approaches and concepts previously explored, such as hierarchical CAs (HCAs) and neural CAs to create hierarchical neural cellular automata, and we evolve their behavior to grow into and maintain a target shape, thus selecting for morphogenesis and homeostasis. We found that such HNCAs are more evolvable (given a morphogenesis and homeostasis task) than their non-hierarchical counterparts. This result and its implications are explored below.